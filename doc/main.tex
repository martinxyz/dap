\title{Desktop Awareness Project}
\author{
        Martin Renold \\
        Personal Research
        }
\date{\today}

% Title reasoning...
%Desktop Awareness Project (DAP)                      
%- who is aware? the users about the desktop? nope... 
%User Awareness Project (UAP)                         
%                                                     
%Desktop Sensor System (DSS) 
%Desktop Sensing System

\documentclass[]{article}

\begin{document}
\maketitle

\section{Introduction}

These days Desktop environments are mostly thought of as a container
where the user can run independent applications. No program cares what
other applications are running, what dialog the user currently
interacts, whether the user is anywhere near the keyboard at all, and
whether the sound has been muted.

The goal of this project is to constantly watch what the user is doing
with the desktop, and expose that information to application
developers. During the project, some sample applications should be
developed, eg. giving the user a statistic about what program is being
interacted with how long.

\section{Data Sources}

The idea is to just throw as many different data sources as possible
at the central algorithm, and let it figure out whether they are
useful and how they are correlated. It is more important to have some
data from every source than to have it all processed properly.

\subsection{Keyboard and Mouse}

Keyboard statistics, maybe even some common keyboard shortcuts or
commands (careful about sensitive data...). Mouse motion in pixels
during the last 5 minutes, number of keys pressed during the lats
10seconds, minimum/maximum pause between keys/mouse.

\subsection{Microphone}
When available, the microphone will record many potentially useful
events. It can hear the user approaching and detect whether there is
anybody out there at all. It might even be able to distinguish the
sound of different doors and windows being opened or closed, and it
will most certainly also hear typing and clicking. It will also detect
chairs moving, vacuum cleaner, toilet flushing, cooking activity,
light switches being pressed, electrical disturbances, power supply
noise, music (possibly produced by the host PC), and many things you
would never have thought of.

Some signal processing is required to detect such events. As a first
approach, just calculate the average amplitude over some time, and its
derivative. Later, cepstral coefficients migth come handy.

\subsection{Video}
Video input, if available, produces a huge amount of data to process,
and a lot of literature to read about processing algorithms.  I
suggest to just average the brightness over four screen regions and
record their absolute and relative value and its derivative. This
should at least allow to detect when someone is passing by or moving.

There are tons of other features that could be added later, eg. the
noise level, color, sharpness, etc. and even the output of complete
object recognition algorithms (eg. viola jones face detection).

\subsection{System Statistics}
CPU load, harddisk activity, maybe CPU and Harddisk temperature.  New
processes created, process CPU and memory usage. System uptime, last
reboot, uptime last week, network connectivity, usual poweroff times,
etc.

\subsection{Window Manager}
Ask the Window manager about visible window, window with input focus,
when a window is created, etc.

\subsection{USB Bus}
Plugging of hardware devices. (eg. camera, usb sticks, etc)

\subsection{Audio Output}
Sounds being played by the PC. Beeps and music.

\subsection{Filenames}
A bit trickier: record whenever a file is saved. (To allow the user)


\subsection{Specific Application Events}
Receiving a new email, sending one. URL/Title of the webpage being
surfed to. Person/Channel being chated with/in.

\section{Possible Applications}

\subsection{Application Usage Statistics}
Programs for that already exist on Windows. Display how many hours
each application was used. Can be useful eg. to monitor game
addiction.

Additionally, show an uptime graph with 24 hours on the x-axis and
number of days that the computer was up on the y-axis. Can do same
thing for individual applications.

\subsection{Chat Programs}
For chat programs, usually something like ``mark me idle after 5
minutes of inactivity'' is used. This could be improved by adding an
estimate about the return time, or eg. immediatly marking the user as
unavailable when the door to the room is closed, and mark the user as
available again when he enters the room (even without touching mouse
or keyboard). Privacy is a concern here, of course; just make sure the
user can choose exactly how much information he is giving away.

Also, chat programs often need to draw attention when a new message
comes in, eg. by playing a sound or focusing the window. The acitivity
monitor could be used to delay such things until the next natural
microbreak that the user takes, or the notification delayed until the
user is back.

\subsection{User defined application starters}
If the recognized events can somehow be visualized to the user, the
use can be allowed to attach something to those events, eg. lock the
screen whenever the user stands up, or print a notification if there
was no vacum cleaner sound for more than two weeks, or shut down the
PC when the user left the building.

Extending the concept, the system might even detect repeated sequences
of actions and offer to automate the task. For example when the user
always opens the image in GIMP after taking a screenshot, the system
could ask whether it should do so automatically every time. Or focus
the email window whenever the user comes back.

\subsection{Background Tasks}
An API could be offered that finds out suitable start times for
starting background task like backups or indexing.

\subsection{Memory Refreshers}

[This might not belong into this project.]

Get memory refreshers about specific stuff, eg. if you view an email
of a person that you rarely read emails from, offer to view the
webpage where you last read about that same person. Same thing with
rare topics...

\section{Architecture}
hm...

\end{document}

